\cleardoublepage
\thispagestyle{plain}

% Inhaltsverzeichnis-Eintrag
\titlecontents{chapter}
[0em]
{\vspace{12pt}}
{\contentslabel{1em}}
{}
{\vzPunkte\contentspage}

% Code für Linksbuendigkeit der Klammern 
\makeatletter 
	\renewcommand*{\@biblabel}[1]{\makebox[\labelwidth][l]{[#1]}}
\makeatother

% Label-Width - Eintrag entspricht max. Anzahl an Ziffern
\setbiblabelwidth{99}

% Literaturverzeichnis Stil
% 2023/06/23: changes bibliography style to a English one
\bibliographystyle{ieeetr}
%	Pfad:		C:\Program Files (x86)\MiKTeX 2.9\miktex\bin\bibtex.exe
%	Argumente:	"%bm"}
% Am besten aus Citavi exportieren
\bibliography{Verzeichnis_Literatur}


%%%%%% Extra Verzeichnisse %%%%%
\cleardoublepage
\thispagestyle{plain}

% Inhaltsverzeichnis-Eintrag
\titlecontents{chapter}
[0em]
{}
{\contentslabel{1em}}
{}
{\vzPunkte\contentspage}

%%%%% Quellenverzeichnis %%%%%%

% Label-Width - Eintrag entspricht max. Anzahl an Ziffern
\setbiblabelwidth{9}


%	Postprozessor:
		%Anwendung: 	bibtex.exe
		%Argumente:		Q



%% Layout Inhaltsverzeichnis wiederherstellen
%\input{Format_ToC}